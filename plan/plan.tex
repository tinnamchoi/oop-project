\documentclass{article}

\usepackage[a4paper]{geometry}
\usepackage{lipsum}
\usepackage{graphicx}
\usepackage{amsmath}

\title{Project plan}
\author{Darcy Geyer, Sarah McCauley, Tin Nam Choi}
\date{2022\\October}

\begin{document}
  \maketitle{}
  
  \tableofcontents{}
  \setlength{\parindent}{0em}
  \setlength{\parskip}{1em}
  
  \pagebreak
  
  \section{Description}
  The game is a text-based, turn-based, player-vs-CPU fighting and adventure game. Initially, the player chooses tocatch 3 Pokemon from a given list of several Pokemon, each with unique details and stats (name, type, level, HP, basicattack, special attack, and defense). It is explained to the player that their objective is to reach a final landmark (the topof Mount Fuji), and that to get there, they will encounter enemies (CPUs), each holding 1 to 3 Pokemon. The player willhave to battle and defeat 10 enemies, each being more difficult than the last. After defeating the 3rd, 6th, and 9thenemy, the player will encounter a new landmark, which will act as a checkpoint. If the player’s pokemon are alldefeated in a battle, the player will be returned to the most recently passed checkpoint, and if any enemies weredefeated after passing the checkpoint, they will be reset, and the player will have to battle them again. The player winsthe game once they have defeated all 10 enemies, and reached the top of Mount Fuji.
  
  \subsection{Pokemon}
  \begin{itemize}
    \item Name: Each Pokemon has a unique name. 
    \item Type: Each Pokemon possesses a type (either water, fire, or grass).
      \begin{itemize}
        \item Water is weak to grass attacks
        \item 
      \end{itemize}
  \end{itemize}
  
  \pagebreak

  \section{Potential classes}
  
  \subsection{Player}
  Data: 
  \begin{itemize}
    \item Pokemon owned
    \item Level 
  \end{itemize}
  Functions:
  \begin{itemize}
    \item Add/remove Pokemon owned
    \item Increment/decrement level
  \end{itemize}
  
  \subsubsection{Person}
  Data:
  \begin{itemize}
    \item Name
    \item Skill points (for leveling up)
  \end{itemize}
  Functions:
  \begin{itemize}
    \item Set name
    \item Increment/decrement skill points
    \item Get action from user
  \end{itemize}
  
  \subsubsection{Computer}
  Functions:
  \begin{itemize}
    \item Get action randomly
  \end{itemize}
    
  \subsection{Pokemon}
  Data:
  \begin{itemize}
    \item Name
    \item Type
    \item Level
    \item Health
    \item Attack
    \item Defense
    \item Speed
    \item Moves learnt
  \end{itemize}
  Functions:
  \begin{itemize}
    \item Increment/decrement level
    \item Increment/decrement health
    \item Increment/decrement attack
    \item Increment/decrement defense
    \item Increment/decrement speed
    \item Add/remove moves learnt
  \end{itemize}
  
  \subsection{Menu}
  Data:
  \begin{itemize}
    \item Title
    \item Options (vector)
  \end{itemize}
  Functions:
  \begin{itemize}
    \item Print menu
    \item Set title
    \item Set options
  \end{itemize}
  
  \pagebreak
  
  \section{Timeline}
  
  \subsection*{Mid-term break}
  \begin{itemize}
    \item Finalize plan
    \item Prototypical testing
  \end{itemize}
  
  \subsection*{Week 9}
  \begin{itemize}
    \item Submit plan
    \item Begin coding
  \end{itemize}
  
  \subsection*{Week 10}
  \begin{itemize}
    \item Finish version 1
  \end{itemize}
  
  \subsection*{Week 11}
  \begin{itemize}
    \item Finish coding
  \end{itemize}
  
  \pagebreak
  
  \section{User interface}
  
  The game will use a command-line interface. This will be done using the Menu class to display the options to the user, and the user will be able to select an option by typing the corresponding number. 
  
  \begin{center}\includegraphics[height=2cm]{media/Menu.png}\end{center}
  
  \section{Unit testing and debugging}
    
\end{document}
